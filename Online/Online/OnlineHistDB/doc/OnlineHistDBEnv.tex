\item   typedef enum { H1D, H2D, P1D, P2D, CNT, SAM} HistType;
\item    inline int {\bf debug}() const ;

 get verbosity level (0 for no debug messages, up to 3)


\item    void {\bf setDebug}(int DebugLevel) ;

 set verbosity level


\item    int {\bf excLevel}() const ;

 exception level: 0 means never throw exc. to client code, 1 means only
 severe errors (default value), 2 means throw all exceptions.
 The default value is 1, meaning that exceptions are thrown only in
 case of severe DB inconsistency. All other errors, e.g. syntax errors,
 can be checked from the method return values and from the warning
 messages on the standard output.
 Exceptions can be catched using {\it catch(std::string message)}


\item    void {\bf setExcLevel}(int ExceptionLevel) ;




\item    std::string {\bf PagenameSyntax}(std::string fullname,\\\mbox{}~~~~~~~~~ std::string \&folder);


 check the syntax of the page full name, returning the correct syntax and the folder name 


\item    int {\bf getAlgorithmNpar}(std::string AlgName,\\\mbox{}~~~~~~~~~
		       int* Ninput = NULL);

 gets the number of parameters, and optionally the number of input histograms, needed by algorithm AlgName.


\item    std::string {\bf getAlgParName}(std::string AlgName,\\\mbox{}~~~~~~~~~
			    int Ipar);

 gets the name of parameter Ipar (starting from 1) of algorithm AlgName


\item    inline std::string {\bf refRoot}() ;

 get reference histograms root directory 


\item    void {\bf setRefRoot}(std::string newroot) ;

 change reference histograms root directory (for testing)


